\chapter{Capítol 1. Matrius i Vectors} \label{cap:matrius}

\section{Resum}

\section{Exercicis i problemes} \label{sec:matrius-exercicis}
\begin{exercise}[1.8.1]\label{ex:1.8.1}

Sigui $\alpha \in \mathbb{R}$. Demostreu que la potència k-èsima de la matriu:

\begin{equation}
\label{e181_A}
A = \begin{pmatrix}
    \cos(\alpha) & -\sin(\alpha)\\
    \sin(\alpha) & \cos(\alpha)\\
\end{pmatrix}
\end{equation}

és la matriu:


\begin{equation} 
\label{e181_Ak}
A^k = \begin{pmatrix}
    \cos(k\alpha) & -\sin(k\alpha)\\
    \sin(k\alpha) & \cos(k\alpha)\\
\end{pmatrix}
\end{equation}
\end{exercise}


\begin{solution}
Veiem-ho per inducció completa. És evident que pel cas $n=1$ es compleix.

Veiem que si es compleix per la potència $k$-èsima, llavors es compleix per la potència $k+1$-èsima.
\begin{align}
    A^{k+1}= A^k\cdot A &= 
    \begin{pmatrix}
            \cos(k\alpha) & -\sin(k\alpha)\\
    \sin(k\alpha) & \cos(k\alpha)\\
    \end{pmatrix}\cdot \begin{pmatrix}
    \cos(\alpha) & -\sin(\alpha)\\
    \sin(\alpha) & \cos(\alpha)\\ 
\end{pmatrix},\\
&=\begin{pmatrix}
    \cos(k\alpha) \cos(\alpha)-\sin(k\alpha)\sin(\alpha) & -\cos(k\alpha)\sin(\alpha)-\sin(k\alpha)\cos(\alpha)\\
\sin(k\alpha)\cos(\alpha)+\cos(k\alpha)\sin(\alpha) & -\sin(k\alpha) \sin(\alpha)+\cos(k\alpha)\cos(\alpha)\\
\end{pmatrix}
\end{align}

Fem servir les fórmules d’addició:
\[
\cos(x+y)=\cos x\cos y-\sin x\sin y,
\qquad
\sin(x+y)=\sin x\cos y+\cos x\sin y.
\]
Amb $x=k\alpha$ i $y=\alpha$, obtenim
\[
A^{k+1}=
\begin{pmatrix}
\cos((k+1)\alpha) & -\sin((k+1)\alpha)\\
\sin((k+1)\alpha) & \cos((k+1)\alpha)
\end{pmatrix},
\]
i queda provat el pas inductiu.

\end{solution}
\newpage
\begin{exercise}[1.8.2]
Calculeu totes les potència $A^k$ de la matriu quadrada d'ordre $n$, $A=(a^j_i)$, que té tots els coeficients zero llevats dels situats just sobre la diagonal. En altres paraules,

\[
a_{i,j} = \left\{\begin{aligned}
    &1, \text{ si } i = j+1,\\
    &0, \text{ si } i \neq j+1.
\end{aligned}\right.
\]

\textit{Indicació.} Feu-ho primer per a $n=2,3...$. Conjectureu què passarà en general i demostreu-ho de la manera següent: suposeu que és cert per a un cert $n=k$ i proveu-ho per a $n=k+1$ calculant $A^{k+1} = A\cdot A^k.$
\end{exercise}

\begin{solution}
   Observem primer el patró que apareix en calcular les potències de $A$.

   \begin{align} A^{2} = A\cdot A &= 
    \begin{pmatrix} 0 & 1 & 0 &\hdots&0 &0 &0 \\ 
        0 & 0 &1 &\hdots &0 &0 &0 \\ 
        0 & 0 &0 &\hdots &0 &0 &0 \\ 
        \vdots & \vdots &\vdots &\ddots &\vdots &\vdots &\vdots \\ 
        0 & 0 & 0 &\hdots & 0&1 &0 \\ 0 & 0 & 0 &\hdots & 0&0 &1 \\ 
        0 & 0 &0 &\hdots &0 &0 &0 
    \end{pmatrix}\cdot 
    \begin{pmatrix} 
        0 & 1 & 0 &\hdots&0 &0 &0 \\ 
        0 & 0 &1 &\hdots &0 &0 &0 \\ 
        0 & 0 &0 &\hdots &0 &0 &0 \\ 
        \vdots & \vdots &\vdots &\ddots &\vdots &\vdots &\vdots \\ 
        0 & 0 & 0 &\hdots & 0&1 &0 \\ 
        0 & 0 & 0 &\hdots & 0&0 &1 \\ 
        0 & 0 &0 &\hdots &0 &0 &0 
    \end{pmatrix},\\ 
    &=\begin{pmatrix} 
        0 & 0 & 1 &\hdots&0 &0 &0 \\ 
        0 & 0 &0 &\hdots &0 &0 &0 \\ 
        0 & 0 &0 &\hdots &0 &0 &0 \\ 
        \vdots & \vdots &\vdots &\ddots &\vdots &\vdots &\vdots \\ 
        0 & 0 & 0 &\hdots & 0&0 &1 \\ 
        0 & 0 & 0 &\hdots & 0&0 &0 \\ 
        0 & 0 &0 &\hdots &0 &0 &0 \end{pmatrix}\\ 
        &= \left\{ \begin{aligned} 
            &1, \text{ si } i = j+2,\\ 
            &0, \text{ si } i \neq j+2. 
        \end{aligned}\right. 
    \end{align}

    És a dir, els uns apareixen sobre la diagonal definida per $i=j+2$, que és la diagonal que comença a la tercera columna.

    La nostra hipòtesi (que evidentment es compleix per als casos $k=1$ i $k=2$) és que la matriu $A^k$ estarà definida com 
    \begin{equation}
        \label{ex182Ap}
        (A^p)_{ij} = \delta_{i,j+p},
    \end{equation}
    on $\delta_{ij}$ denota el \emph{delta de Kronecker}, recordem:
    \[
    \delta_{ij} =
    \begin{cases}
    1, & \text{si } i = j,\\
    0, & \text{si } i \neq j.
    \end{cases}
    \]
    Suposem que es compleix per al cas $k=p$ i veiem que es compleix per a $k=p+1$.
    
    A partir de \eqref{ex182Ap}, el pas inductiu es verifica de manera immediata. En efecte,
    \[
    (A^{p+1})_{ij}
    = \sum_{k=1}^n (A^p)_{ik}\,A_{kj}
    = \sum_{k=1}^n \delta_{i,k+p}\,\delta_{k,j+1}
    = \delta_{i,j+p+1},
    \]
    Quan escrivim el producte:
    \[
    \delta_{i,k+p}\,\delta_{k,j+1},
    \]
    estem indicant que aquest val $1$ si i només si es compleixen simultàniament les dues condicions:
    \[
    \begin{cases}
    i = k + p,\\
    k = j + 1.
    \end{cases}
    \]
    Si alguna d’aquestes condicions falla, el producte és nul. En aquest sentit, el producte de deltes de Kronecker representa la intersecció de les condicions imposades sobre els índexs.

    Substituint la segona condició a la primera, obtenim
    \[
    i = k + p = (j+1) + p = j + (p+1).
    \]
    Per tant, les dues condicions conjuntes són equivalents a una sola relació,
    \[
    i = j + p + 1.
    \]
    I amb aquest argument demostrem que el patró es conserva per a $A^{p+1}$.


\end{solution}